\documentclass[]{article}

\usepackage{latexsym}
\usepackage{amssymb}
\usepackage{amsmath}

%opening
\title{Homework 1}
\author{Ida Hönigmann}

\begin{document}

\maketitle

\section*{Problem 1}

\begin{align*}
	p(t)=det(A-t \cdot \textrm{Id}) =
	\begin{array}{|cccc|}
		a_{11}-t & a_{12} & \cdots & a_{1n} \\
		a_{21} & a_{22} - t & \cdots & a_{2n} \\
		\vdots & \vdots & & \vdots \\
		a_{n1}-t & a_{n2} & \cdots & a_{nn} - t \\
	\end{array}
\end{align*}


\section*{Problem 2}

The Gamma function is defined as

\begin{align*}
	\Gamma (x) := \lim\limits_{n \to \infty} \frac{n! n^x}{x(x+1)\cdots (x+n)}
\end{align*}

\noindent
There holds the Weierstraß product representation

\begin{align*}
	\frac{1}{\Gamma (x)} = x \cdot e^{Cx} \cdot \prod_{k=1}^{\infty} (1 + \frac{x}{k}) e^{-k/k} && \textrm{with} && C := \lim\limits_{n \to \infty} (\sum_{k=1}^{n} \frac{1}{k} - \ln n)
\end{align*}


\section*{Problem 3}

Let $f,g : \mathbb{R} \to \mathbb{R}$ be given functions given by

\begin{align*}
	f(x) := \left\{
	\begin{array}{ll}
		-1 & \textrm{if}\ x < - \frac{\pi}{2}, \\
		\sin (x) & \textrm{if}\ - \frac{\pi}{2} \leq x \leq \frac{\pi}{2}, \\
		1 & \textrm{if}\ x > \frac{\pi}{2}. \\
	\end{array}
	\right. & \textrm{and} & 
	g(x) := \left\{
	\begin{array}{ll}
		1 & \textrm{if}\ x \in \mathbb{Q}, \\
		0 & \textrm{if}\ x \in \mathbb{R} \setminus \mathbb{Q}. \\
	\end{array}
	\right.
\end{align*}

\section*{Problem 4}

For $q \in \mathbb{R}$, it holds that

\begin{align*}
	\lim\limits_{n \to \infty} q^n = 
	\left\{
	\begin{array}{ll}
		+\infty & \textrm{if}\ q > 1, \\
		1 & \textrm{if}\ q = 1, \\
		0 & \textrm{if}\ -1 < q < 1, \\
		\nexists & \textrm{if}\ q \leq -1. \\
	\end{array}
	\right.
\end{align*}


\section*{Problem 5}

\begin{align*}
	A := \left(
	\begin{array}{ccccc}
		\alpha & 2\alpha & 3\alpha & \cdots & n\alpha \\
		0      & \alpha  & 2\alpha & \ddots & \vdots  \\
		0      & 0       & \ddots  & \ddots & 3\alpha \\
		\vdots & \ddots  & \ddots  & \ddots & 2\alpha \\
		0      & \cdots  & 0       & 0      & \alpha  \\
	\end{array}
	\right) \in \mathbb{R}^{n \times n}_{\textrm{tria}}
\end{align*}

\end{document}
